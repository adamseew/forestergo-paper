
\newcommand{\CLASSINPUTtoptextmargin}{54pt}
\newcommand{\CLASSINPUTbottomtextmargin}{54pt}
\newcommand{\CLASSINPUTinnersidemargin}{54pt}
\newcommand{\CLASSINPUToutersidemargin}{54pt}

\documentclass[letterpaper,10pt,conference,twoside]{IEEEtran}
\IEEEoverridecommandlockouts 
\def\IEEEtitletopspace{21pt}

\usepackage[inline]{enumitem}
\usepackage{amsmath}
\usepackage{amsfonts}
\usepackage{amssymb}
\usepackage{algorithmic}
\usepackage{algorithm}
\usepackage{array}
\usepackage{textcomp}
\usepackage{stfloats}
\usepackage{verbatim}
\usepackage{graphicx}
\usepackage[font=footnotesize]{caption}
\usepackage[font=footnotesize]{subcaption}
\usepackage[noadjust]{cite}

%% package for urls
\usepackage{url}

%% hyperref
% and an override to make hyperref work with ieeetran.cls
\makeatletter
\let\NAT@parse\undefined
\makeatother
\usepackage[pagebackref=true,breaklinks=true,colorlinks,bookmarks=false]{hyperref}
\makeatletter
\newcommand*{\textlabel}[2]{%
  \edef\@currentlabel{#1}% Set target label
  \phantomsection% Correct hyper reference link
  #1\label{#2}% Print and store label
}
\makeatother

\usepackage{cleveref}[2012/02/15]% v0.18.4; 
\crefformat{footnote}{#2\footnotemark[#1]#3}
\usepackage{textpos}
\usepackage{amsthm}
\usepackage{xcolor}
\usepackage{tikz}
\usepackage[scaled]{helvet}
\usepackage{flushend}


\AddToHook{shipout/foreground}{
  \begin{tikzpicture}[remember picture,overlay]
    \node[red,rotate=45,scale=10,opacity=0.2] at (current page.center) {\small\fontfamily{phv}\selectfont%};
     IN~PREPARATION};
%    UNDER~REVIEW};   
  \end{tikzpicture}
}

%% correct bad hyphenation here
\hyphenation{}

\renewcommand{\qedsymbol}{$\blacksquare$}

\theoremstyle{definition}
\newtheorem{thm}{Theorem}[section]
\newtheorem{lem}[thm]{Lemma}
\newtheorem{prop}[thm]{Proposition}
\newtheorem{assm}[thm]{Assumption}
\newtheorem{cor}{Corollary}
\newtheorem{conj}{Conjecture}[section]
\newtheorem{defn}{Definition}[section]
\newtheorem{exmp}{Example}[section]
\newtheorem{pb}{Problem}[section]
\newtheorem{rem}{Remark}
\newtheorem{obs}{Observation}
\newtheorem*{ctb}{Contribution}

\DeclareMathOperator*{\argmax}{arg\,max}
\DeclareMathOperator*{\argmin}{arg\,min}

\makeatletter
\newcommand\notsotiny{\@setfontsize\notsotiny\@vipt\@viipt}
\makeatother

\renewcommand\citepunct{,\hspace*{.8ex}}
\renewcommand*{\citedash}{--}

\begin{document}
\bstctlcite{IEEEexample:BSTcontrol}

\title{\LARGE\bf Scaling Ergodic Control for Large-Scale Problems:\\Intra-Forest Exploration with a Moving Gaussian Mixture Model}

\author{Adam Seewald${}^{\text{1}}$, Ian Abraham${}^{\text{2}}$, and Stefano Mintchev${}^{\text{1}}$
  \thanks{This work was partly supported by ETH Z{\"u}rich's World Food System Center and Yale University.}
  \thanks{${}^{\text{1}}$A.\hspace*{.4ex}S. and S.\hspace*{.4ex}M. are with the Department of Environmental Systems Science, ETH Z{\"u}rich, Switzerland. Email: {\tt\footnotesize \href{mailto:aseewald@ethz.ch}{aseewald@ethz.ch};}}
  \thanks{${}^{\text{2}}$I.\hspace*{.4ex}A. is with the Department of Mechanical Engineering and Materials Science, Yale University, CT, USA.}
}

\maketitle

\vspace*{-.5cm}
\begin{abstract}
\end{abstract}



%%%%%%%%%%%%%%%%%%%%%%
\section{Introduction}
\noindent
% story line, ergodic search is great because blah blah
% there has been a lot of developments recently, but
% no one is really scaling ergodic control to real and large-scale pb.
% here we do that with the concept of a moving GMM
% moreover, existing approaches rely on external techniques for obstacle avoidance (e.g., the cbfs Cameron's been using)
% here we propose a a methodology that has a degree of obstacle avoidance as a "by-product."
% BUT WHY IS ERGO EXCITING?
% Deep fundamental challenges linking machine learning, optimal control, signal processing and information theory
% Achieves manipulation tasks robustly, by not only relying on accurate sensors, but instead using a control strategy to cope with limited or inaccurate sensing information
% Different from stochastic or patterned search! → Provides a natural way of searching
Exploring unknown and potentially large-scale spaces with robots is a problem commonly addressed by different methodologies arising in computing and robotics. This problem is recurring in real-world use case such as monitoring, reconstruction, exploration, etc., where robots are expected to cover a given space while performing an assigned task. A key challenge to the practical applicability of these methodologies is that of leveraging resources, while, at the same time, maximizing the information gathered and optimizing the exploration accordingly~\cite{popovic2020informative,schmid2020efficient}. 
While there are different approaches in the literature, approacheas that are informed by sensory data have emerged as of particular interest. Among these approaches, ergodic control is a significant result, as it provides a more natural way of serching through deterministic exploratory behaviours while accounting for both the motion cost and optimality~\cite{miller2016ergodic}.

Ergodic control is a planning and controls methodology that derives robot trajecotries maximizing a given information measure, so that robots spend more time in areas with high information measure while quickly traversing areas with low information measure~\cite{mathew2011metrics,abraham2017ergodic,miller2013trajectory}. As a consequence, it is required that the user provides an information measure a priori, or that, the information measure is derived as the robots gather more information about their sourroundings. 
While applicable to some use cases, this is often a limitation of the existing ergodic controllers. It is indeed not always the case the information measure can be easily refined from the gathered data, or, that an a-priori information measure can be provided at all. With this work we addressed this challenge. We provide a general and scaled ergodic control methodology that can be applied to a brother class of robotic use cases at large-scale. Our methodology does not require an underlying information measure but rather derives an information measure from the exploration itself, and refines such measure utilizing information about already visited areas and obstacles from sensory data. 
Furthermore, conversely to existing ergodic control methodologies that require extenrnal obstacle avoidance techniques, e.g., control barrier functions~\cite{lerch2023safety}, optical flow~\cite{prabhakar2020ergodic}, etc., our methodology provides a tunable degree of obstacle avoidance as a ``by-product.'' 
%
The underlying information measure is represented utilizing a Gaussian Mixture Model (GMM), which is refined -- a process that is handled by our methodology and that does not require any user input -- from the sensory data as the robot traverses the state space. Our ergodic formulation is different from existing methods. The problem is posed so that the robot spends time in areas with low infromation measure, whereas the ``explored space'' is related to high information measure. 
The methodology is iterative and utilizes a model predictive controller (MPC) approach, where the ergodic controller is refined within a specified time window. 

Existing ergodic controllers have been studied from the point of view of time~\cite{dong2023time} and energy~\cite{seewald2024energy,naveed2024eclares} optimality and applied to a multitude of use cases. These use cases include tactile sensing~\cite{abraham2017ergodic}, active learning~\cite{abraham2021ergodic}, multi-objective optimality~\cite{ren2023pareto,srinivasan2023multi}, grasping and manipulation~\cite{shetty2022ergodic,bilaloglu2023whole} and visual rendering~\cite{low2022drozbot,prabhakar2020autonomous}. Ergodic controllers in the literature feature diverse aspects such as stochastic dynamics~\cite{torre2016ergodic,ayvali2017ergodic} and multi-agent and/or swarm control with both centralized~\cite{seewald2024energy,rao2024learning} and distributed information processing~\cite{prabhakar2020ergodic,coffin2022multi}. Although some of these use-cases involve information gathering~\cite{dressel2018optimality} and features urban environments and other potentially large-scale problems~\cite{prabhakar2020ergodic,rao2023multi}, a generic large-scale ergodic controller has not been studied yet.
Even though recent methods have been making progress in this direction~\cite{whittemeyer2023bi,seewald2024energy,naveed2024eclares,dong2023time}, these methods do not scale efficiently due to the formulation of the underlying optimization and/or require an a-priori information measure. From an obstacle avoidance perspective, even though there are ergodic controllers that feature obstacle avoidance~\cite{lerch2023safety}, this is an external component on top of the ergodic controller that then results in a sub-ergodic solution~\cite{dong2023time}, rather then an integral component of the explorer itself.

We show case our general and scaled ergodic control on a large-scale problem: the problem of exploring a forest or a densely vegetated area, in contrast to existing ergodic control methodologies that are demonstrated in structured environments, in simulation, or generally in small-scale only. Section~\ref{sec:res} shows the performance of our methods with both simulated and real-world data. 
The open-source software stack to replicate our approach is made availabel on our project repository webpage\footnote{.}.

The remainder of this paper is then structured as follows. Sec.~\ref{sec:pb} provides the details of the underlying principles and the problem addressed. Sec.~\ref{sec:meth} is split into two sub-sections, one provides an overview of ergodic control whereas the other one details our methodology. Sec.~\ref{sec:conc} provides conclusins and draws future perspectives.

%%%%%%%%%%%%%%%%%%%%%%%%%%%%%
\section{Problem Formulation}\label{sec:pb}
%\IEEEpubidadjcol
\noindent
This work addresses the problem of exploring a bounded and potentially large-scale space, whereby large-scale we indicate spaces of orders of dozens or even hundred of meters in both x- and y-axis. For practical reasons, we bound the exploration space, i.e., the space to be explored, to one hectare, and we condsider the exploration in two-dimensions. The formulation is such, however, that the state space might be potentially unbounded, and not limited to two dimensions~\cite{dong2023time}.

Let us thus consider such bounded space $\mathcal{Q}\subset\mathbb{R}^2$. The robot is free to move in the space except for a finite number of obstacles represented by $\mathcal{O}\subset\mathcal{Q}$.  
In the remainder, we utilize the concepts of ergodicity and ergodic metric to direct the robot into unexplored areas while avoiding the obstacles, i.e., $\mathcal{Q}\cap\mathcal{O}$, as opposed to other works on ergodic control in the literature where the exploration happens in areas of high information density instead~\cite{mathew2011metrics,abraham2017ergodic,miller2013trajectory}.
\begin{defn}[Ergodicity]
  Given the bounded state space $\mathcal{Q}$, a trajectory $\mathbf{x}(t)\in\mathcal{Q}$ is \textit{ergodic} w.r.t. a spatial distribution $\phi$, or, analogously, is distributed among regions of high expected distribution, if and only if
  \begin{equation}
    \lim_{t\rightarrow\infty}{\int_{\mathcal{Q}}\phi(\mathbf{x})\Omega(\mathbf{x})\,d\mathbf{x}=\frac{1}{t}\int_{\mathcal{T}}{{\Omega\big(\overline{\mathbf{x}}(t)\big)}}\,dt},
  \end{equation}
  where $\overline{\,\cdot\,}$ is a map that maps the state space to the exploration spaces, and $\Omega$ are all the Lebesgue function as defined in, e.g.,~\cite{TODO}.
\end{defn}

The spatial distribution $\phi$ is built utilizing a Gaussian Mixture Model (GMM). 
\begin{defn}[Moving GMM]\label{def:movement}
  Assume that there is a given number $n\in\mathbb{N}_{>0}$ of Gaussians $\mathcal{N}$ in a GMM, whose initial probability is equally distributed. A \textit{moving GMM} is
  \begin{equation}
    \phi(\boldsymbol{\alpha},\boldsymbol{\mu},\mathbf{x}):=\sum_{i=1}^n{\alpha_i\,\mathcal{N}_i(\mathbf{x}\,|\,\mu_i,\Sigma_i)},
  \end{equation}
  where $\Sigma_i\in\mathbb{R}^{2\times 2}$ indicates the the covariance matrix and $\mu_i\in\mathcal{Q}$ the center of a Gaussian $\mathcal{N}_i$, i.e., a GMM with variable centers $\boldsymbol{\mu}\in\mathcal{Q}^n$ and mixing coefficients $\boldsymbol{\alpha}\in\mathbb{R}_{>0}^n$.
  
  Bold letters indicate vector, i.e., $\mathbf{x}\in\mathcal{Q}$ is the state space vector whereas $\boldsymbol{\alpha}$, $\boldsymbol{\mu}$ are the vectors compromising ideal GMM's probability and position components respectively (see Section~\ref{sec:meth}).
   
\end{defn}

An ergodic metric is a value that quantifies the ergodicity.
\begin{defn}[Ergodic metric]\label{def:ergomet}
  Consider a time average distribution of the trajecotry over a limited time window $t$, e.g.,
  \begin{equation}
    h\big(\mathbf{x}(t)\big):=\frac{1}{t}\int_\mathcal{T}\Delta\big((\mathbf{x})(t)\big)\,dt.
  \end{equation}
  $\Delta$ is defined as a Dirac delta function. An \textit{ergodic metric} is an $L^2$-inner prodact in between the average of the spatial and time distributions.
\end{defn}

\begin{pb}[Scaled ergodic control]\label{pb}
  Given the state space and the obstacles space $\mathcal{Q}$ and $\mathcal{O}$ respectively, assume the number of components $n$ of a GMM is given. The \textit{scaled ergodic control} problem is the problem of finding the evolution of the components of the moving GMM, i.e., $\boldsymbol{\alpha}(t)$ and $\boldsymbol{\mu}(t)$ and of the control $\mathbf{u}(t)\in\mathcal{U}$ so that $\mathbf{x}(t)$ explores $\mathcal{Q}$ while avoiding $\mathcal{O}$ \textit{and}, the ergodic metric is minimized.
\end{pb}

We propose a solution to Problem~\ref{pb} and demostrate both by simulation and real-world experiment the feasibility of the solution, underlying the existence of trade-offs in between the accuracy and the exploration soundness in Sec.~\ref{sec:res} and~\ref{sec:res} respectively.

%%%%%%%%%%%%%%%%%%
\section{Methods}\label{sec:meth}
\noindent
This section details our methods. Sec.~\ref{sec:canon} introduces the concept of canonical ergodic control for exploration in bounded areas with an information density distribution. Sec.~\ref{sec:sol} describes our methodology of scaled ergodic control, i.e., ergodic control with a moving information density as a function of explored against unexplored space.

\subsection{Canonical ergodic control}
\label{sec:canon}
\noindent
To quantify the time average and the average of the spatial distributions $h$ and $\phi$ respectively, let us consider Fourier series basis functions~\cite{TODO}. For the time average distribtuion, the coefficients of an equivalent basis function can be expressed
\begin{equation}\label{eq:ck}
  c_k\big(\mathbf{x}(t)\big):=\int_{\mathcal{T}}{\prod_{d\in\{1,2\}}}{\cos{\big(2\pi k_d\mathbf{x}_d(\tau)/T\big)}/T^2}\,d\tau/t,
\end{equation}
where $T\in\mathbb{R}_{>0}$ is a given period. $\,\cdot\,_d$ indicates the $d$th item of a vector. %, so that $d$ iterates all the dimensions.

Equation~(\ref{eq:ck}) expresses the cosine basis function for a coefficient $k$, i.e., we consider only the positive slice of the spectral domain and thus ignore the function's imaginary component. The coefficients $k\in\mathcal{K}$ depend on a given number of frequencies $\kappa\in\mathbb{N}_{>0}$ and are built so that $\mathcal{K}\in\mathbb{N}^2$ is a set of index vectors that cover the set $\kappa\times\cdots\times\kappa\in\mathbb{N}^{\kappa^2}$, i.e., the coefficients are evaluated on the entire domain.

For the average of the spatial distribution, the coefficients of an equivalent basis function can be expressed similarly
\begin{equation}\label{eq:phik}
  \phi_k(\mathbf{x}):=\int_{\mathcal{Q}}{\sum_{d\in\{1,2\}}}{\phi(\mathbf{x})c(\mathbf{x})\,d\mathbf{x}},
\end{equation}
where $c$ is the integrand in Eq.~(\ref{eq:ck}) in the given point $\mathbf{x}$ at the current time step.

The aim of an ergodic controller is to minimize an ergodic metric, i.e., the $L^2$-inner product of the distributions $h$ and $\phi$ (see Definition~\ref{def:ergomet}). A consolidated metric~\cite{abraham2017ergodic,abraham2021ergodic,seewald2024energy,lerch2023safety,abraham2018decentralized,dong2023time} for this purpose is, for instance
\begin{equation}
  \mathcal{E}(\mathbf{x}):=\sum_{k\in\mathcal{K}}{\Lambda_k(c_k-\phi_k)^2/2},
\end{equation}
where the coefficients of the time average and the average spatial distributions are expressed in Eq.~(\ref{eq:ck}--\ref{eq:phik}). $\Lambda_k$ is a weight factor that expresses which frequency has more weight, e.g., with %the factor
\begin{equation}
  \Lambda_k:=\frac{1}{\sqrt{\big(1+\lVert{k}\rVert^2\big)^3}},
\end{equation}
lower frequencies are to be preferred.

Note that in Eq.~(\ref{eq:phik}) we have utilized the expression for a standard GMM. % instead of a moving GMM in Definition~\ref{def:movement}. 
We utilize the expression of the moving GMM in the next section.

\subsection{Scaled ergodic control}
\label{sec:sol}
\noindent
To utilize the concept of moving information density as a function of explored against unexplored space, let us first consider Eq.~(\ref{eq:phik}) with the moving GMM from Definition~\ref{def:movement}.

Let us assume for practical purposes that the space is square, with a given length $l\in\mathbb{R}_{>0}$ expressed in meters.
Let us thus tight the period in Eq.~(\ref{eq:ck}) to such search space and express $l=T/2$. Eq.~\ref{eq:phik} can be expressed
\begin{equation}
  \phi_k(\boldsymbol{\alpha},\boldsymbol{\mu},\mathbf{x})\hspace*{-.6ex}:=\hspace*{-1ex}\int_{\mathcal{Q}}{\hspace*{-1ex}\Bigg(\hspace*{-.3ex}\sum_{d\in\{1,2\}}\hspace*{-.5ex}\sum_{i=1}^n\alpha_i\mathcal{N}_i\big(\mathbf{x\,|\,\overline{\rule{0pt}{2.4mm}{\mu_i}},\overline{\Sigma_i}}\big)\hspace*{-.5ex}\Bigg)\hspace*{-.1ex}c(\mathbf{x})\,d\mathbf{x}},
\end{equation}
where $\overline{\,\cdot\,}$ is map that maps the center and the covariance matrix to a symmetric state space delimited by $-l$ and $l$ by, e.g., using linear transformation matrices~\cite{calinon2020mixture}.

Let us further define a given value that expresses the concept of ``history.'' If this value is expressed by, $h\in\mathbb{R}_{>0}$, we can model the concept of the ``already explored space'' exploting the definition of the moving GMM. The covariance matrix can be expressed
\begin{equation}
  \Sigma_i:=\frac{1}{2T}\int_\Upsilon\sum_{d\in\{1,2\}}{\big(\mathbf{x}_d(\tau)-\mu_i\big)}\,d\tau,
\end{equation}
where the trajectory is being evaluated within the history, i.e., $\Upsilon$ indicates the time interval in betweeen $t$ and $t-h$.
%
The centers can be then expressed as $\mu_i:=E\big(\mathbf{x}(t)\big)$ with $E$ being the expected value of $\mathbf{x}$ on $\Upsilon$. 

The scaled ergodic control problem can then be expressed as the problem of finding an ergodic controller that visit the inverse of the probability distribution represented by the moving GMM, thus avoiding areas ``already visited,'' within a given history window $h$. The problem posed in this way, however, produces trajectories that require an additional obstacle avoidance methodology, such as~\cite{lerch2023safety}.

Let us thus consider a modified expression for the center of the gaussian $\mu_i:=E\big(\mathbf{x}(t)\big)-e_i$ where $e_i\in\mathbf{e}\subset\mathbb{R}^n$ is a displacement that allow to ``move'' the Gaussian components in the moving GMM. 

Our methodology is such that the scaled ergodic controller finds the minimum displacement of the Gaussians so that the the space to be visited is delimited by $\mathcal{Q}\cap\mathcal{O}$. Such controller can be expressed with the optimal control problem (OCP)
\begin{subequations}\label{eq:ocp}\begin{align}
  \min_{\Theta}%\int_{\mathcal{T}}\mathbf{u}(\tau)^TR\mathbf{u}(\tau)\,d\tau+
  \mathcal{E}&(\mathbf{x})+\Psi(\mathbf{e}),\label{eq:cost}\\
  \text{s.t. }\dot{\mathbf{x}}&=f(\mathbf{x}(t),\mathbf{u}(t)),\label{eq:dyn}\\
  \mathbf{x}&(t)\in\mathcal{Q}\cap\mathcal{O},\,\mathbf{u}(t)\in\mathcal{U},\\
  \mathbf{x}&(t_0),n,\kappa,l,h,\text{ are given},\label{eq:ocpconsttotf}
\end{align}\end{subequations}
where the output of the optimization $\Theta$ in Eq.~(\ref{eq:cost}) is $\mathbf{x},\mathbf{u},\boldsymbol{\alpha},\boldsymbol{\mu}$, i.e., the center of each Gaussian and its mixing coefficient in the moving GMM along the control and the state. The function $\Psi$ maps the displacement to a cost value, e.g., $\Sigma_{e_i\in\mathbf{e}}|e_i|$, where $|\hspace{.2ex}\cdot\hspace{.2ex}|$ is the defined as an $L^2$-norm.

The dynamics in Eq.~(\ref{eq:dyn}) is a 2D single integrator system, which mimics the behaviour of a real UAV in our experimental setup to a reasonable extent (see Sec.~\ref{sec:res}).

The problem is formulated to find the optimal control, displacement, and probability for each Gaussian, ensuring that the displacement and probability deviate minimally from the ideal case, n.b., here the Gaussians represent the history of the explored space. For practical purposes, a given horizon $N \in \mathbb{N}_{>0}$ is defined, and the optimization is reiterated for each horizon using a methodology similar to an MPC controller, i.e., $\mathcal{T}$ in Eq.~(\ref{eq:ck}) is the interval in between $t$ and $t-N$.

The large-scale exploration is considered concluded when a desired level of coverage is achieved (see Sec.~\ref{sec:res}). It is also possible to set up the problem so that the exploration does not terminate or lasts for long-term, such as in~\cite{seewald2022energy}.

Other practical considerations, such as the choice of the history and the size of the moving GMM, are detailed in the next section.


\begin{figure}[t!]
  \begin{minipage}[t!]{.25\columnwidth}
    \caption[.]{\textbf{.   }.   .   .   .   .   .   .   .   .   .   .   .   .   .   .   .   .   .   .   .   .   .   .   .   .   .   .   .   .   .   .   .   .   .   .   .   .   .   .   .   .   .   .   .   .   .   .   .   .   .   .   .   .   .   .   .   .   .   .   .   .   .   .   .   .   .   .   .   .   .   .   .   .   .   .   .   .   .   .   .   .   .   .   .   .   .   .   .   .   .   .   .   .   .   .   .   .   .   .   .   .   .   .   .   .   .   .   .   .   .   .   .   .   .   .   .   .   .   .   .   .   .   .   .   .   .   .   .   .   .   .   .   .   .   .   .   .   .   .   .   .   .   .   .   .   .   .   .   .   .   .   .   .   .   .   .   .   .   .   .   .   .   .   .   .   .   .   .   .   .   .   .   .   .   .   .   .   .   .   .   .   .   .   .   .   .   .   .   .   .   .   .   .   .   .   .   .   .   .   .   .   .   .   .   .   .   .   .   .   .   .   .   .   .   .   .   .   .   .   .   .   .   .   .   .   .   .   .   .   .   .   .   .   .   .   .   .   .   .   .   .   .   .   .   .   .   .   .   .   .   .   .   .   .   .   .   .   . }
      %
  \end{minipage}\hspace*{.3cm}
  \begin{minipage}[t!]{.7\columnwidth}
    \vspace*{-.3cm}
    \input{figures/compare.pdf_tex}
  \end{minipage}
  \vspace*{-.4cm}
  \caption*{\footnotesize .   .   .   .   .   .   .   .   .   .   .   .   .   .   .   .   .   .   .   .   .   .   .   .   .   .   .   .   .   .   .   .   .   .   .   .   .   .   .   .   .   .   .   .   .   .   .   .   .   .   .   .   .   .   .   .   .   .   .   .   .   .   .   .   .   .   .   .   .   .   .   .   .   .   .   .   .   .   .   .   .   .   .   .   .   .   .   .   .   .   .   .   .   .   .   .   .   .   .   .   .   .   .   .   .   .   .   .   .}
  \label{fig:0}
\end{figure}






%%%%%%%%%%%%%%%%%%%%%%%%%%%%%%
\section{Experimental Results}\label{sec:res}
\noindent
This section provides an overview of our experimental setup and showcases the results. Simulated experiments are implemented utilizing \textsc{Matlab} (R), whereas physical experimental setup is aided with a rutine that utilizes Python and conducted with the popular DJI (R) Mavic 3 Classic Unmanned Aerial Vehicle (UAV). The exploratory trajectories are imported into the UAV's flight controller utilizing waypoints, which is handled by a proprietary software component; but portability with other software components and flight controllers is supported (see Sec.~\ref{sec:conc}). 
The MPC optimization stack, i.e., the solver that solves the OCP in Eq.~(\ref{eq:ocp}), relies on two external open-source components. The non-linear programming solver IPOPT~\cite{wachter2006implementation} and the algorithmic differentiation library CasADi~\cite{andersson2012casadi}.

In the following, Sec.~\ref{sec:res1} describes simulated experiemtnal results, a forest with an area of 3 600 squared meteres. Sec.~\ref{sec:res2} details our finding in terms of built-in obstacle avoidance capabilities against coverage of our general scaled ergodic controller. Sec.~\ref{sec:res3} describes our real-world experimental results, a vegetated area with multiple obstacles  with an area of 10 000 squared meters.

\subsection{Simulated forest}\label{sec:res1}
\noindent
.

\subsection{Obstacle avoidance vs. coverage}\label{sec:res2}
\noindent
.

\subsection{Real-world experiments}\label{sec:res3}
\noindent
.



~
\newpage

\begin{figure}[t!]
  \centering
  \vspace*{-.3cm}
  \begin{subfigure}[t]{\linewidth}
  \hspace*{-.15cm}\input{figures/__steps_1.pdf_tex}
  \caption{\textbf{.   }.   .   .   .   .   .   .   .   .   .   .   .   .   .   .   .   .   .   .   .   .   .   .   .   .   .   .   .   .   .   .   .   .   .   .   .   .   .   .   .   .   .   .   .   .   .   .   .   .   .   .   .   .   .   .   .   .   .   .   .   .   .   .   .   .   .   .   .   .   .   .   .   .   .   .   .   .   .   .   .   .   .   .   .   .   .   .   .   .   .   .   .   .   .   .   .   .   .   .   .   .   .}
  \end{subfigure}
  \begin{subfigure}[t]{\linewidth}
  \hspace*{-.15cm}\input{figures/_steps_2.pdf_tex}
  \caption{\textbf{.   }.   .   .   .   .   .   .   .   .   .   .   .   .   .   .   .   .   .   .   .   .   .   .   .   .   .   .   .   .   .   .   .   .   .   .   .   .   .   .   .   .   .   .   .   .   .   .   .   .   .   .   .   .   .   .   .   .   .   .   .   .   .   .   .   .   .   .   .   .   .   .   .   .   .   .   .   .   .   .   .   .   .   .   .   .   .   .   .   .   .   .   .   .   .   .   .   .   .   .   .   .   .}
  \end{subfigure}
  \begin{subfigure}[t]{\linewidth}
  \hspace*{-.15cm}\input{figures/_steps_3.pdf_tex}
  \caption{\textbf{.   }.   .   .   .   .   .   .   .   .   .   .   .   .   .   .   .   .   .   .   .   .   .   .   .   .   .   .   .   .   .   .   .   .   .   .   .   .   .   .   .   .   .   .   .   .   .   .   .   .   .   .   .}
  \end{subfigure}
  \caption[.]{\textbf{.   }.   .   .   .   .   .   .   .   .   .   .   .   .   .   .   .   .   .   .   .   .   .   .   .   .   .   .   .   .   .   .   .   .   .   .   .   .   .   .   .   .   .   .   .   .   .   .   .   .   .   .   .   .   .   .   .   .   .   .   .   .   .   .   .   .   .   .   .   .   .   .   .   .   .   .   .   .   .   .   .   .   .   .   .   .   .   .   .   .   .   .   .   .   .   .   .   .   .   .   .   .   .   .   .   .   .   .   .   .   .   .   .   .   .   .   .   .   .   .   .   .   .   .   .   .   .   .   .   .   .   .   .   .   .   .   .   .   .   .   .   .   .   .   .   .   .   .   .   .   .   .   .}
  \label{fig:1}
\end{figure}




%%%%%%%%%%%%%%%%%%%%%%%%%%%%%%%%%%%%%%%%%%
\section{Conclusion and Future Directions}\label{sec:conc}
\noindent
.

{\small
\bibliographystyle{IEEEtran} 
\bibliography{forestergo}
}

\end{document}

